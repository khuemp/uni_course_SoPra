\documentclass[10pt,a4paper]{article}

\usepackage[utf8]{inputenc}
\usepackage{amsmath}
\usepackage{amsfonts}
\usepackage{amssymb}
\usepackage[margin=0.75in]{geometry}
\usepackage[backgroundcolor=blue!20!white]{todonotes}
\usepackage{titling}
%\setlength{\droptitle}{-10em}
\begin{document}
\title{SoPra - \textbf{Projektbeschreibung}\\ - Einzelphase -}
\renewcommand\maketitlehookc{\vspace{-9ex}}
\maketitle

\begin{center}
    Max Musterfrau \textbf{0000001}
\end{center}

\todo[inline,backgroundcolor=red]{Dieses Dokument erklärt oberflächlich den Aufbau des Servers der Referenzimplementierung aus
der Gruppenphase. In Ihrer Abgabe müssen Sie die Funktionsweise des Servers nicht genauer erklären, solange Sie auf unserer
Implementierung aufbauen. Sie sollten jedoch möglichst detailliert die eingeführten Änderungen und deren Funktionsweise erklären.
Je ausführlicher Ihre Erklärung ist, desto besser können wir Ihren Entwurf bewerten. Dies kann im Zweifel entscheidend für das
Bestehen des Software-Praktikums sein.}

\section*{Funktionsweise}
\todo[inline]{Hier beschreiben Sie, wie Ihr Server funktioniert und wie Sie die Änderungen umgesetzt haben.}
\noindent Das Grundprinzip, nach dem wir unsere Implementierung entworfen haben, war \emph{MVC}. Wir haben also versucht, Logik und Daten voneinander
abzutrennen. Um eben dies zu erreichen, nutzen wir eine Command- sowie zwei Visitor-Strukturen, welche die Spiellogik
kapseln. Davon abgegrenzt sind unsere Datenklassen. Diese enthalten (fast) keine Logik, sondern nur Daten. Eine Ausnahme hierzu sind
erweiterte Getter und Setter, welche den Spielzustand intern konsistent halten sollen.\\
Falls die Spielerin nun ein Command an den Server sendet, so wird dieser direkt ausgeführt. Hierbei operiert der Command ausschließlich
auf unserem Model und bekommt (zum Senden von Events) noch einen \linebreak \texttt{ConnectionWrapper}, welcher die \texttt{ServerConnection} wrapped.
Für den Fall, dass ein Command ausgeführt wird, welcher von einer Ability beeinflusst wird, so wird ein entsprechender \texttt{AbilityVisitor}
erstellt und der Visitor übernimmt die entsprechende Logik des eingegangenen Commands. Es gibt außerdem eine separate Visitor-Struktur,
um das Nutzen von Karten abzuhandeln. Unsere Abilities sind als \emph{Decorator} modelliert.\\
Ebenfalls nutzen wir einen Builder und Factories, um das Einlesen und Validieren der Konfigurationsdatei möglichst gut von unserer eigentlichen
Modelstruktur abzukapseln.


\section*{Designentscheidungen}
\todo[inline]{Beschreiben Sie hier genau, wie Sie die bestehende Implementierung geändert haben und begründen Sie Ihre Entscheidungen.}
\noindent Wir haben uns hauptsächlich für drei Patterns entschieden: \emph{Builder}, \emph{Visitor} und \emph{Command}.
\begin{itemize}
    \item \textbf{Builder}: Die Nutzung eines Builders erlaubt es uns, die Erstellung der Entities und der Welt zu kapseln.
                            Ebenfalls haben wir uns dazu entschieden, das Parsen und Validieren so modular zu halten, dass
                            es prinzipiell möglich wäre, eine andere Modelstruktur damit aufzubauen.
    \item \textbf{Visitor}: Die Überlebenden haben verschiedene Fähigkeiten und können durch das Ausrüsten bestimmter Karten weitere
                            Fähigkeiten hinzugewinnen. Um auf die richtige Fähigkeit korrekt zu reagiern, müssen wir wissen um welche
                            es sich handelt. Dies realisieren wir durch die entsprechenden Visitors.
    \item \textbf{Command}: Das Commandpattern bietet sich bei diesem Projekt gut an, da es die Interaktion mit der \texttt{ServerConnection}
                            vorsieht, Commands zu verwenden. Wir nutzen die Commands, um die Befehle der Spieler auszuführen und die
                            Logik der Befehle vom Rest des Spiels abzukapseln.
\end{itemize}

\section*{Herausforderungen}
\todo[inline]{Welche Änderungen waren am schwierigsten umzusetzen? Wie haben Sie die daraus resultierenden Probleme gelöst?}
\noindent Die größte Herausforderung bei diesem Projekt war es, das Handhaben der Fähigkeiten und Ausrüstungskarten zu modellieren ohne die Spiellogik überall zu
verteilen, bzw. ohne mit \texttt{instanceof} o.Ä. herauszufinden mit welchen Abilities man gerade arbeitet. Die Nutzung der Visitors hilft hierbei.

\end{document}
